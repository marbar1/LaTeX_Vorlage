%%%%%%%%%%%%%%%%%%%%%%%%%%%%%%%%%%%%%%%%%%%%%%%%%%%%%%%%%%%%%%%%%%%%%%%%%%%%%%%%%%%%%%%%%%%%%%%%%%%%
%                                                                                                  %
%                                              MASTER                                              %
%                                                                                                  %
%%%%%%%%%%%%%%%%%%%%%%%%%%%%%%%%%%%%%%%%%%%%%%%%%%%%%%%%%%%%%%%%%%%%%%%%%%%%%%%%%%%%%%%%%%%%%%%%%%%%


\documentclass{Studienarbeit}

%--------------------------------------------------------------------------------------------------
% REQUIRED PACKAGES
%--------------------------------------------------------------------------------------------------
\usepackage{graphicx} % Include images in the document
\usepackage{placeins} % FloatBarrier command
\usepackage{appendix} % Appendices at the end of the document.
\usepackage[export]{adjustbox} % adjust and resize boxes/ alignment options for includegraphics
\usepackage{fancyhdr} % Custom header and footer
\usepackage{lastpage} % Reference the number of pages is your document
\usepackage[german]{babel} % Manages culturally-dertermined typographical rules
\usepackage{csquotes} % Advanced facilities for inline and display quotation
\usepackage{tocbibind} % Disables the inclusion of the ToC [nottoc]
\usepackage[hidelinks]{hyperref}% use option [hidelinks] to hide red boxes around links in PDF
\usepackage[
acronym, % acronyms and glossary
automake=immediate, % automake=immediate to be able to print the glossary https://tex.stackexchange.com/questions/468818/glossaries-automake-not-working-lualatex
toc,
section=section,
nopostdot]{glossaries} % create glossary and acronyms
\usepackage[
  backend=biber,
  bibencoding=utf8,
  sorting=none
]{biblatex} % bibliography


%--------------------------------------------------------------------------------------------------
% DEFINITIONS
%--------------------------------------------------------------------------------------------------
\graphicspath{{figures/}}
\renewcommand{\title}{Titel}
\renewcommand{\author}{Autorname}
\newcommand{\projectname}{Projektname}
\newcommand{\courseofstudies}{Studiengang}
\newcommand{\universityname}{Name der Universität}
\newcommand{\processingperiod}{Bearbeitungszeitraum}
\newcommand{\matnumber}{Matrikelnummer}
\newcommand{\courseabbreviation}{Kursabkürzung}
\newcommand{\companyname}{Firmenname}
\newcommand{\companycity}{Standort der Firma}
\newcommand{\supervisorcompany}{Betreuer der Firma}
\newcommand{\supervisordhbw}{Gutachter der DHBW}
\newcommand{\location}{Ort}
\newcolumntype{R}{>{\raggedright\let\newline\\\arraybackslash\hspace{0pt}}X}


%--------------------------------------------------------------------------------------------------
% PAGE STYLE
%--------------------------------------------------------------------------------------------------
\renewcommand{\floatpagefraction}{.8}

\fancypagestyle{zpagestyle}{ 
\renewcommand{\headrulewidth}{0.5pt}
\renewcommand{\footrulewidth}{0pt}

% delete old header and footer
\fancyhead{}
\fancyfoot{}

% left header
\fancyhead[L]{
\nouppercase{\leftmark}
}

% center header
\fancyhead[C]{
\center
\vspace*{2mm}
}

% right header
\fancyhead[R]{
\includegraphics[height=0.8cm]{figures/left_logo.pdf}
\hspace{0.4cm}
\includegraphics[height=0.8cm]{figures/right_logo.pdf}
}
%--------------------------------------------------------------------------------------------------
% left footer
\fancyfoot[L]{
}

% center footer
\fancyfoot[C]{
{\thepage} von \pageref{LastPage}
}

% right footer
\fancyfoot[R]{
}
}
\pagestyle{zpagestyle}%


%--------------------------------------------------------------------------------------------------
% GLOSSARY
%--------------------------------------------------------------------------------------------------
\makeglossaries
\loadglsentries{Glossar}


%--------------------------------------------------------------------------------------------------
% BIBLIOGRAPHY
%--------------------------------------------------------------------------------------------------
\addbibresource{Literatur.bib}

% Break URL in biblatex at lowercase and uppercase characters 
% https://tex.stackexchange.com/questions/134191/line-breaks-of-long-urls-in-biblatex-bibliography
\setcounter{biburllcpenalty}{7000}
\setcounter{biburlucpenalty}{8000}

% Make supercite use square brackets
% From: https://tex.stackexchange.com/a/60923
\DeclareCiteCommand{\supercite}[\mkbibsuperscript]{%
    \iffieldundef{prenote}
    {}
    {\BibliographyWarning{Ignoring prenote argument}}%
    \iffieldundef{postnote}
    {}
    {}
  }
  {\usebibmacro{citeindex}%
   \bibopenbracket%
   \usebibmacro{cite}%
   \usebibmacro{postnote}%
   \bibclosebracket}
  {\supercitedelim}
  {}


%--------------------------------------------------------------------------------------------------
% BEGIN DOCUMENT
%--------------------------------------------------------------------------------------------------
\begin{document}

% TITLEPAGE
\begin{titlepage}
  \newcommand{\HRule}{\rule{\linewidth}{0.5mm}}
  \begin{center}
  \vspace*{-2cm}
  \begin{figure}[!ht]
  \centering
  \begin{minipage}{.5\textwidth}
      \includegraphics[width=.5\linewidth, left]{figures/left_logo.pdf}
  \end{minipage}%
  \begin{minipage}{.5\textwidth}
      \includegraphics[width=.5\linewidth, right]{figures/right_logo.pdf}
  \end{minipage}
  \end{figure}
  \vspace*{1cm}
  
  \HRule \\[0.4cm]
  \textsc{\Huge
  \title
  }\\[1mm]
  \HRule \\[1.2cm]
  {\LARGE \projectname} \\[1.2cm]
  \begin{large}
  des Studienganges \courseofstudies \\
  an der \universityname \\[0.5cm]
  von \\
  \author \\[1cm]
  \todayshort \\[1cm]
  \end{large}
  
  \begin{large}
  \begin{tabular}{ l l }
      Bearbeitungszeitraum            & {\processingperiod} \\
      Matrikelnummer, Kurs            & {\matnumber}, {\courseabbreviation} \\
      Ausbildungsfirma                & {\companyname}, {\companycity} \\
      Betreuer der Ausbildungsfirma   & {\supervisorcompany} \\
      Gutachter der Dualen Hochschule & {\supervisordhbw}
  \end{tabular}
  \end{large}
  \end{center}
\end{titlepage}

% SPERRVERMERK
\newpage
\input{Sperrvermerk}

% EIGENSTÄNDIGKEITERKLÄRUNG
\input{Eigenstaendigkeitserklaerung}

% ABSTRACT
\include{Abstract}

% INHALTSVERZEICHNIS
\setcounter{tocdepth}{2}
\tableofcontents
\newpage

% ABBILDUNGSVERZEICHNIS
\listoffigures

% TABELLENVERZEICHNIS
\listoftables

% QUELLCODEVERZEICHNIS
\listoflistings

% AKRONYME
\clearpage
\printglossary[type=\acronymtype]

% GLOSSAR
\newpage
\printglossary[type=main]

% HAUPTTEIL
%%%%%%%%%%%%%%%%%%%%%%%%%%%%%%%%%%%%%%%%%%%%%%%%%%%%%%%%%%%%%%%%%%%%%%%%%%%%%%%%%%%%%%%%%%%%%%%%%%%%
%                                                                                                  %
%                                             MAIN PART                                            %
%                                                                                                  %
%%%%%%%%%%%%%%%%%%%%%%%%%%%%%%%%%%%%%%%%%%%%%%%%%%%%%%%%%%%%%%%%%%%%%%%%%%%%%%%%%%%%%%%%%%%%%%%%%%%%

%--------------------------------------------------------------------------------------------------
% Einleitung
%--------------------------------------------------------------------------------------------------
\section{Einleitung}
\label{sec:intro}
Hier steht Text mit einer Abkürzung \gls{ack} und einer Quellenangabe. \supercite{bosch:cantiming}

% LITERATURVERZEICHNIS
\newpage
\printbibliography[heading=bibintoc,title={Literaturverzeichnis}]

% ANHÄNGE
\include{Anhaenge}

\end{document}