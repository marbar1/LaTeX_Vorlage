% useful: https://tex.stackexchange.com/questions/489732/how-to-draw-curly-braces-on-minted-and-tcolorbox

\documentclass[10pt, a4paper]{Studienarbeit}

\usepackage[german]{babel}
\usepackage{csquotes}
\usepackage[acronym, 
automake=immediate, toc, section=section]{glossaries}
% automake=immediate to be able to print the glossary https://tex.stackexchange.com/questions/468818/glossaries-automake-not-working-lualatex
\usepackage{minted}
\usepackage{etoolbox}
\usepackage{tcolorbox}
\usepackage{tikz}
\usetikzlibrary{positioning}
\usepackage{graphicx}
\usepackage{hyperref}% use option [hidelinks] to hide red boxes around links in PDF
%\BeforeBeginEnvironment{minted}{\begin{tcolorbox}}%
%\AfterEndEnvironment{minted}{\end{tcolorbox}}%

%--------------------------------------------------------------------------------------------------
% DEFINITIONS
%--------------------------------------------------------------------------------------------------
\graphicspath{{figures/}}
\title{Titel}
\author{Autorname}
\date{\todayshort}
\newcommand{\projectname}{Projektname}
\renewcommand{\doctitle}{\projectname}


% set the emergencystrech to 3 em in case a long word or a mintinline Element can be linebroken and a Badbox Message occurs. em stands for the fontsize so for example if the fontsize is 12 pt (12 pt = approx. 4.233mm) 3em would be 12.7mm
% more info: https://github.com/gpoore/minted/issues/31
\setlength{\emergencystretch}{3em}

%\glossarystyle{altlist} % list, altlist, listgroup, listhypergroup
\makeglossaries

\addbibresource{Literatur.bib}

% very useful: https://tex.stackexchange.com/questions/8946/how-to-combine-acronym-and-glossary
% put "\hfill \\" at the beginnig of a description to start the definition at a new line

\makeglossaries
\renewcommand{\glossaryentrynumbers}[1]{(#1)}

% Example for (acronym), (acronym and glossary) and glossary
%--------------------------------------------------------------------------------------------------
% Computer-aided design (CAD)
%--------------------------------------------------------------------------------------------------
\newglossaryentry{cad}{
	type=\acronymtype,
	name={CAD},
	description={\textbf{C}omputer-\textbf{a}ided \textbf{d}esign},
	first={Computer-aided design (CAD)}
}


%--------------------------------------------------------------------------------------------------
% Human Machine Interface (HMI)
%--------------------------------------------------------------------------------------------------
\newglossaryentry{glos:hmi}{
	name={HMI},
	description={Ein HMI, \textbf{H}uman-\textbf{M}achine \textbf{I}nterface (dt. Mensch-Maschine Schnittstelle), ist eine Schnittstelle zwischen dem Benutzer und einer Maschine und ermöglicht die Kommunikation mit dem Gerät. Es kann sich dabei um Hard- oder Software handeln. Im Rahmen dieses Berichts bezieht sich die {\dq}Zimmer-HMI{\dq} auf eine Software, mit der Daten von Greifern gelesen werden sowie Greifer gesteuert werden können}
}

\newglossaryentry{hmi}{
	type=\acronymtype,
	name={HMI\glsadd{glos:hmi}},
	description={\textbf{H}uman-\textbf{M}achine \textbf{I}nterface},
	first={Human-Machine Interface (HMI)\glsadd{glos:hmi}},
	see=[Glossar:]{glos:hmi}
}


%--------------------------------------------------------------------------------------------------
% LuaLaTeX
%--------------------------------------------------------------------------------------------------
\newglossaryentry{glos:lualatex}{
	name={Lua\LaTeX},
	description={Lua{\LaTeX} basiert auf Lua{\TeX}...}
}



%--------------------------------------------------------------------------------------------------
% BEGIN DOCUMENT
%--------------------------------------------------------------------------------------------------
\begin{document}

\maketitlepage

% To skip a page if it happens to be a even number
\emptycleardoublepage
%%%%%%%%%%%%%%%%%%%%%%%%%%%%%%%%%%%%%%%%%%%%%%%%%%%%%%%%%%%%%%%%%%%%%%%%%%%%%%%%%%%%%%%%%%%%%%%%%%%%
%                                                                                                  %
%                                     INDEPENDENCE DECLARATION                                     %
%                                                                                                  %
%%%%%%%%%%%%%%%%%%%%%%%%%%%%%%%%%%%%%%%%%%%%%%%%%%%%%%%%%%%%%%%%%%%%%%%%%%%%%%%%%%%%%%%%%%%%%%%%%%%%

% Der Text für die Erklärung wurde aus den Richtlinien für Studienarbeiten der DHBW entnommen:
% https://www.dhbw.de/fileadmin/user_upload/Dokumente/Dokumente_fuer_Studierende/191212_Leitlinien_Praxismodule_Studien_Bachelorarbeiten.pdf
% Stand: Dezember 2019
%
% Der korrekte Ort muss vor der Abgabe/ dem Ausdrucken eingetragen werden.


{\Large \textbf{Erkärung}}
\makeatletter

Ich versichere hiermit, dass ich meine {\@title} mit dem Thema: {\projectname} selbstständig verfasst und keine anderen als die angegebenen Quellen und Hilfsmittel benutzt habe. Ich versichere zudem, dass die eingereichte elektronische Fassung mit der gedruckten Fassung übereinstimmt.

\vspace{1cm}

Ort, den {\@date}

\vspace{2.5cm}

\rule{5cm}{0.4pt}

\@author

\makeatother

% table of contents
\emptycleardoublepage
\tableofcontents

\clearpage
\printglossary[type=\acronymtype]

\newpage
\printglossary[type=main]

\section{Firmenvorstellung}
\label{sec:company}
Text Text Text

\listoffigures

\newpage

\printbibliography[heading=bibintoc,title={Literaturverzeichnis}]

\end{document}
