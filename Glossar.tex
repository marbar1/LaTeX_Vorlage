%%%%%%%%%%%%%%%%%%%%%%%%%%%%%%%%%%%%%%%%%%%%%%%%%%%%%%%%%%%%%%%%%%%%%%%%%%%%%%%%%%%%%%%%%%%%%%%%%%%%
%                                                                                                  %
%                                             GLOSSARY                                             %
%                                                                                                  %
%%%%%%%%%%%%%%%%%%%%%%%%%%%%%%%%%%%%%%%%%%%%%%%%%%%%%%%%%%%%%%%%%%%%%%%%%%%%%%%%%%%%%%%%%%%%%%%%%%%%

% Hier sind die Akronyme, Glossareinträge sowie beides zumsammen definiert.
%
% Tipp:
% put "\hfill \\" at the beginnig of a description to start the definition at a new line

\renewcommand{\glossaryentrynumbers}[1]{(#1)}

%%%%%%%%%%%%%%%%%%%%%%%%%%%%%%%%%%%%%%%%%%%%%%%%%%%%%%%%%%%%%%%%%%%%%%%%%%%%%%%%%%%%%%%%%%%%%%%%%%%%
%                                                                                                  %
%                                             ACRONYMS                                             %
%                                                                                                  %
%%%%%%%%%%%%%%%%%%%%%%%%%%%%%%%%%%%%%%%%%%%%%%%%%%%%%%%%%%%%%%%%%%%%%%%%%%%%%%%%%%%%%%%%%%%%%%%%%%%%


%---------------------------------------------------------------------------------------------------
% Acknowledgement (ACK)
%---------------------------------------------------------------------------------------------------
\newglossaryentry{ack}{
	type=\acronymtype,
	name={ACK},
	description={\textbf{Ack}nowledgement.},
	first={Acknowledgement (ACK)},
}

%---------------------------------------------------------------------------------------------------
% Baud-Rate-Prescaler (BRP)
%---------------------------------------------------------------------------------------------------
\newglossaryentry{brp}{
	type=\acronymtype,
	name={BRP},
	description={\textbf{B}aud-\textbf{R}ate-\textbf{P}rescaler.},
	first={Baud-Rate-Prescaler (BRP)},
}

%---------------------------------------------------------------------------------------------------
% Board Support Package (BSP)
%---------------------------------------------------------------------------------------------------
\newglossaryentry{bsp}{
	type=\acronymtype,
	name={BSP},
	description={\textbf{B}oard \textbf{S}upport \textbf{P}ackage.},
	first={Board Support Package (BSP)},
}

%---------------------------------------------------------------------------------------------------
% Controller Area Network (CAN)
%---------------------------------------------------------------------------------------------------
\newglossaryentry{can}{
	type=\acronymtype,
	name={CAN},
	description={\textbf{C}ontroller \textbf{A}rea \textbf{N}etwork.},
	first={Controller Area Network (CAN)},
}

%---------------------------------------------------------------------------------------------------
% CAN with Flexible Data-rate (CAN FD)
%---------------------------------------------------------------------------------------------------
\newglossaryentry{canfd}{
	type=\acronymtype,
	name={CAN FD},
	description={\textbf{C}ontroller \textbf{A}rea \textbf{N}etwork with \textbf{F}lexible \textbf{D}ata-rate.},
	first={CAN with Flexible Data-rate (CAN FD)},
}

%---------------------------------------------------------------------------------------------------
% CAN in Automation (CiA)
%---------------------------------------------------------------------------------------------------
\newglossaryentry{cia}{
	type=\acronymtype,
	name={CiA},
	description={\textbf{C}AN \textbf{i}n \textbf{A}utomation.},
	first={CAN in Automation (CiA)},
}

%---------------------------------------------------------------------------------------------------
% Communication Object (COB)
%---------------------------------------------------------------------------------------------------
\newglossaryentry{cob}{
	type=\acronymtype,
	name={COB},
	description={\textbf{C}ommunication \textbf{Ob}ject.},
	first={Communication Object (COB)},
}

%---------------------------------------------------------------------------------------------------
% Cyclic Redundancy Check (CRC)
%---------------------------------------------------------------------------------------------------
\newglossaryentry{crc}{
	type=\acronymtype,
	name={CRC},
	description={\textbf{C}yclic \textbf{R}edundancy \textbf{C}heck.},
	first={Cyclic Redundancy Check (CRC)},
}

%---------------------------------------------------------------------------------------------------
% Data Length Code (DLC)
%---------------------------------------------------------------------------------------------------
\newglossaryentry{dlc}{
	type=\acronymtype,
	name={DLC},
	description={\textbf{D}ata \textbf{L}ength \textbf{C}ode.},
	first={Data Length Code (DLC)},
}

%---------------------------------------------------------------------------------------------------
% End of Frame (EOF)
%---------------------------------------------------------------------------------------------------
\newglossaryentry{eof}{
	type=\acronymtype,
	name={EOF},
	description={\textbf{E}nd \textbf{o}f \textbf{F}rame.},
	first={End of Frame (EOF)},
}

%---------------------------------------------------------------------------------------------------
% First In - First Out (FIFO)
%---------------------------------------------------------------------------------------------------
\newglossaryentry{fifo}{
	type=\acronymtype,
	name={FIFO},
	description={\textbf{F}irst \textbf{I}n - \textbf{F}irst \textbf{O}ut.},
	first={First In - First Out (FIFO)},
	plural={FIFOs},
	firstplural={First In - First Out (FIFOs)}
}

%---------------------------------------------------------------------------------------------------
% Hardware Abstraction Layer (HAL)
%---------------------------------------------------------------------------------------------------
\newglossaryentry{hal}{
	type=\acronymtype,
	name={HAL},
	description={\textbf{H}ardware \textbf{A}bstraction \textbf{L}ayer.},
	first={Hardware Abstraction Layer (HAL)},
}

%---------------------------------------------------------------------------------------------------
% Identifier Extention (IDE)
%---------------------------------------------------------------------------------------------------
\newglossaryentry{ide}{
	type=\acronymtype,
	name={IDE},
	description={\textbf{I}dentifier \textbf{E}xtension.},
	first={Identifier Extention (IDE)},
}

%---------------------------------------------------------------------------------------------------
% Intermission Frame Space (IFS)
%---------------------------------------------------------------------------------------------------
\newglossaryentry{ifs}{
	type=\acronymtype,
	name={IFS},
	description={\textbf{I}nter \textbf{F}rame \textbf{S}pace.},
	first={Intermission Frame Space (IFS)},
}

%---------------------------------------------------------------------------------------------------
% International Organization for Standardization (ISO)
%---------------------------------------------------------------------------------------------------
\newglossaryentry{iso}{
	type=\acronymtype,
	name={ISO},
	description={\textbf{I}nternational \textbf{O}rganization for \textbf{S}tandardization.},
	first={International Organization for Standardization (ISO)},
}

%---------------------------------------------------------------------------------------------------
% Network Management (NMT)
%---------------------------------------------------------------------------------------------------
\newglossaryentry{nmt}{
	type=\acronymtype,
	name={NMT},
	description={\textbf{N}etwork \textbf{M}anagemen\textbf{t}.},
	first={Network Management (NMT)},
}

%---------------------------------------------------------------------------------------------------
% Open Systems Interconnection (OSI)
%---------------------------------------------------------------------------------------------------
\newglossaryentry{osi}{
	type=\acronymtype,
	name={OSI},
	description={\textbf{O}pen \textbf{S}ystems \textbf{I}nterconnection.},
	first={Open Systems Interconnection (OSI)},
}
	
%---------------------------------------------------------------------------------------------------
% Process Data Object (PDO)
%---------------------------------------------------------------------------------------------------
\newglossaryentry{pdo}{
	type=\acronymtype,
	name={PDO},
	description={\textbf{P}rocess \textbf{D}ata \textbf{O}bject.},
	first={Process Data Object (PDO)},
	plural={PDOs},
	firstplural={Process Data Objects (PDOs)}
}

%---------------------------------------------------------------------------------------------------
% Remote Transmission Request (RTR)
%---------------------------------------------------------------------------------------------------
\newglossaryentry{rtr}{
	type=\acronymtype,
	name={RTR},
	description={\textbf{R}emote \textbf{T}ransmission \textbf{R}equest.},
	first={Remote Transmission Request (RTR)},
}

%---------------------------------------------------------------------------------------------------
% Service Data Object (SDO)
%---------------------------------------------------------------------------------------------------
\newglossaryentry{sdo}{
	type=\acronymtype,
	name={SDO},
	description={\textbf{S}ervice \textbf{D}ata \textbf{O}bject.},
	first={Service Data Object (SDO)},
	plural={SDOs},
	firstplural={Service Data Objects (SDOs)}
}

%---------------------------------------------------------------------------------------------------
% (Re-)Synchronization Jump Width (SJW)
%---------------------------------------------------------------------------------------------------
\newglossaryentry{sjw}{
	type=\acronymtype,
	name={SJW},
	description={(Re-)\textbf{S}ynchronization \textbf{J}ump \textbf{W}idth.},
	first={(Re\=/)Synchronization Jump Width (SJW)},
}

%---------------------------------------------------------------------------------------------------
% Start of Frame (SOF)
%---------------------------------------------------------------------------------------------------
\newglossaryentry{sof}{
	type=\acronymtype,
	name={SOF},
	description={\textbf{S}tart \textbf{o}f \textbf{F}rame.},
	first={Start of Frame (SOF)},
}


%%%%%%%%%%%%%%%%%%%%%%%%%%%%%%%%%%%%%%%%%%%%%%%%%%%%%%%%%%%%%%%%%%%%%%%%%%%%%%%%%%%%%%%%%%%%%%%%%%%%
%                                                                                                  %
%                                     ACRONYMS WITH GLOSSARIES                                     %
%                                                                                                  %
%%%%%%%%%%%%%%%%%%%%%%%%%%%%%%%%%%%%%%%%%%%%%%%%%%%%%%%%%%%%%%%%%%%%%%%%%%%%%%%%%%%%%%%%%%%%%%%%%%%%

%---------------------------------------------------------------------------------------------------
% CAN Identifier (CAN-ID)
%---------------------------------------------------------------------------------------------------
\newglossaryentry{glos:canid}{
	name={CAN-ID},
	description={Die CAN-ID ist der 11 Bit (Standard Frame) beziehungsweise 29 Bit (Extended Frame) Identifier einer CAN-Nachricht.}
}

\newglossaryentry{canid}{
	type=\acronymtype,
	name={CAN-ID\glsadd{glos:canid}},
	description={\textbf{CAN} \textbf{Id}entifier.},
	first={CAN Identifier (CAN-ID)\glsadd{glos:canid}},
	plural={CAN-IDs\glsadd{glos:canid}},
	firstplural={CAN Identifiers (CAN-IDs)\glsadd{glos:canid}},
	see=[Glossar:]{glos:canid}
}


%---------------------------------------------------------------------------------------------------
% COB Identifier (COB-ID)
%---------------------------------------------------------------------------------------------------
\newglossaryentry{glos:cobid}{
	name={COB-ID},
	description={Die COB-ID ist der 32 Bit Identifier eines Kommunikationsobjekts und enthält die CAN-ID sowie weitere Kontrollbits.}
}

\newglossaryentry{cobid}{
	type=\acronymtype,
	name={COB-ID\glsadd{glos:cobid}},
	description={\textbf{COB} \textbf{Id}entifier.},
	first={COB-Identifier (COB-ID)\glsadd{glos:cobid}},
	plural={COB-IDs\glsadd{glos:cobid}},
	firstplural={COB-Identifiers (COB-IDs)\glsadd{glos:cobid}},
	see=[Glossar:]{glos:cobid}
}


%---------------------------------------------------------------------------------------------------
% Serial Peripheral Interface (SPI)
%---------------------------------------------------------------------------------------------------
\newglossaryentry{glos:spi}{
	name={SPI},
	description={SPI ist ein synchroner, serieller Bus, welcher zum Verbinden von mehreren Mikrocontrollern oder Mikrocontrollern mit Peripherien entwickelt wurde.}
}

\newglossaryentry{spi}{
	type=\acronymtype,
	name={SPI\glsadd{glos:spi}},
	description={\textbf{S}erial \textbf{P}eripheral \textbf{I}nterface.},
	first={Serial Peripheral Interface (SPI)\glsadd{glos:spi}},
	see=[Glossar:]{glos:spi}
}


%---------------------------------------------------------------------------------------------------
% Physical Layer (PHY)
%---------------------------------------------------------------------------------------------------
\newglossaryentry{glos:phy}{
	name={PHY},
	description={Ein PHY ist ein integrierter Schaltkreis, welcher den Spannungspegel zwischen einem Mikrocontroller und dem Übertragungsmedium wandelt.}
}

\newglossaryentry{phy}{
	type=\acronymtype,
	name={PHY\glsadd{glos:phy}},
	description={\textbf{Phy}sical Layer.},
	first={Physical Layer (PHY)\glsadd{glos:phy}},
	see=[Glossar:]{glos:phy}
}

%%%%%%%%%%%%%%%%%%%%%%%%%%%%%%%%%%%%%%%%%%%%%%%%%%%%%%%%%%%%%%%%%%%%%%%%%%%%%%%%%%%%%%%%%%%%%%%%%%%%
%                                                                                                  %
%                                            GLOSSARIES                                            %
%                                                                                                  %
%%%%%%%%%%%%%%%%%%%%%%%%%%%%%%%%%%%%%%%%%%%%%%%%%%%%%%%%%%%%%%%%%%%%%%%%%%%%%%%%%%%%%%%%%%%%%%%%%%%%

%--------------------------------------------------------------------------------------------------
% Arbitrierung
%--------------------------------------------------------------------------------------------------
\newglossaryentry{glos:arbitrierung}{
    name={Arbitrierung},
    description={Die Arbitrierung ist das Verfahren zur Auswahl eines Gerätes oder Teilnehmers, welchem bei Erfolg Zugriff auf eine gemeinsame Ressource gewährt wird. Im Falle von CAN ist das der Bus.}
}

%--------------------------------------------------------------------------------------------------
% Echtzeit
%--------------------------------------------------------------------------------------------------
\newglossaryentry{glos:echtzeit}{
    name={Echtzeit},
    description={Echtzeit bedeutet die Einhaltung von zuvor definierten Zeitintervallen beim Ausführen von zyklischen Programmen. Ein System wird als echtzeitfähig bezeichnet, wenn es das zuverlässige Ausführen von Funktionen innerhalb einer Zeitgrenze garantieren kann. Dieses Zeitintervall ist nicht im Begriff Echtzeit definiert und kann von einigen Mikrosekunden bis zu Stunden gehen.}
}

%--------------------------------------------------------------------------------------------------
% IO-Link
%--------------------------------------------------------------------------------------------------
\newglossaryentry{glos:iolink}{
	name={IO-Link},
	description={IO-Link ist ein Punkt-zu-Punkt Kommunikationssystem zur Anbindung intelligenter Sensoren und Aktoren. Es ist für kurze Distanzen und geringe Datenmengen gedacht und ermöglicht eine bidirektionale Kommunikation mit Feldgeräten.\supercite{iolink:iolink}}
}


%--------------------------------------------------------------------------------------------------
% Protokollstack
%--------------------------------------------------------------------------------------------------
\newglossaryentry{glos:stack}{
	name={Protokollstack},
	description={Ein Protokollstack oder kurz Stack, ist Software, welche eine Reihe von Protokollen implementiert.}
}