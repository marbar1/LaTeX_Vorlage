%%%%%%%%%%%%%%%%%%%%%%%%%%%%%%%%%%%%%%%%%%%%%%%%%%%%%%%%%%%%%%%%%%%%%%%%%%%%%%%%%%%%%%%%%%%%%%%%%%%%
%                                                                                                  %
%                                             GLOSSARY                                             %
%                                                                                                  %
%%%%%%%%%%%%%%%%%%%%%%%%%%%%%%%%%%%%%%%%%%%%%%%%%%%%%%%%%%%%%%%%%%%%%%%%%%%%%%%%%%%%%%%%%%%%%%%%%%%%

% Hier sind die Akronyme, Glossareinträge sowie beides zumsammen definiert.
%
% Tipp:
% put "\hfill \\" at the beginnig of a description to start the definition at a new line

\renewcommand{\glossaryentrynumbers}[1]{(#1)}

%%%%%%%%%%%%%%%%%%%%%%%%%%%%%%%%%%%%%%%%%%%%%%%%%%%%%%%%%%%%%%%%%%%%%%%%%%%%%%%%%%%%%%%%%%%%%%%%%%%%
%                                                                                                  %
%                                             ACRONYMS                                             %
%                                                                                                  %
%%%%%%%%%%%%%%%%%%%%%%%%%%%%%%%%%%%%%%%%%%%%%%%%%%%%%%%%%%%%%%%%%%%%%%%%%%%%%%%%%%%%%%%%%%%%%%%%%%%%


%---------------------------------------------------------------------------------------------------
% Acknowledgement (ACK)
%---------------------------------------------------------------------------------------------------
\newglossaryentry{ack}{
	type=\acronymtype,
	name={ACK},
	description={\textbf{Ack}nowledgement.},
	first={Acknowledgement (ACK)},
}


%%%%%%%%%%%%%%%%%%%%%%%%%%%%%%%%%%%%%%%%%%%%%%%%%%%%%%%%%%%%%%%%%%%%%%%%%%%%%%%%%%%%%%%%%%%%%%%%%%%%
%                                                                                                  %
%                                     ACRONYMS WITH GLOSSARIES                                     %
%                                                                                                  %
%%%%%%%%%%%%%%%%%%%%%%%%%%%%%%%%%%%%%%%%%%%%%%%%%%%%%%%%%%%%%%%%%%%%%%%%%%%%%%%%%%%%%%%%%%%%%%%%%%%%

%---------------------------------------------------------------------------------------------------
% CAN Identifier (CAN-ID)
%---------------------------------------------------------------------------------------------------
\newglossaryentry{glos:canid}{
	name={CAN-ID},
	description={Die CAN-ID ist der 11 Bit (Standard Frame) beziehungsweise 29 Bit (Extended Frame) Identifier einer CAN-Nachricht.}
}

\newglossaryentry{canid}{
	type=\acronymtype,
	name={CAN-ID\glsadd{glos:canid}},
	description={\textbf{CAN} \textbf{Id}entifier.},
	first={CAN Identifier (CAN-ID)\glsadd{glos:canid}},
	plural={CAN-IDs\glsadd{glos:canid}},
	firstplural={CAN Identifiers (CAN-IDs)\glsadd{glos:canid}},
	see=[Glossar:]{glos:canid}
}


%%%%%%%%%%%%%%%%%%%%%%%%%%%%%%%%%%%%%%%%%%%%%%%%%%%%%%%%%%%%%%%%%%%%%%%%%%%%%%%%%%%%%%%%%%%%%%%%%%%%
%                                                                                                  %
%                                            GLOSSARIES                                            %
%                                                                                                  %
%%%%%%%%%%%%%%%%%%%%%%%%%%%%%%%%%%%%%%%%%%%%%%%%%%%%%%%%%%%%%%%%%%%%%%%%%%%%%%%%%%%%%%%%%%%%%%%%%%%%

%--------------------------------------------------------------------------------------------------
% Arbitrierung
%--------------------------------------------------------------------------------------------------
\newglossaryentry{glos:arbitrierung}{
    name={Arbitrierung},
    description={Die Arbitrierung ist das Verfahren zur Auswahl eines Gerätes oder Teilnehmers, welchem bei Erfolg Zugriff auf eine gemeinsame Ressource gewährt wird. Im Falle von CAN ist das der Bus.}
}